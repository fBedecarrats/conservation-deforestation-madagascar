% Options for packages loaded elsewhere
\PassOptionsToPackage{unicode}{hyperref}
\PassOptionsToPackage{hyphens}{url}
\PassOptionsToPackage{dvipsnames,svgnames,x11names}{xcolor}
%
\documentclass[
  letterpaper,
  DIV=11,
  numbers=noendperiod]{scrreprt}

\usepackage{amsmath,amssymb}
\usepackage{lmodern}
\usepackage{iftex}
\ifPDFTeX
  \usepackage[T1]{fontenc}
  \usepackage[utf8]{inputenc}
  \usepackage{textcomp} % provide euro and other symbols
\else % if luatex or xetex
  \usepackage{unicode-math}
  \defaultfontfeatures{Scale=MatchLowercase}
  \defaultfontfeatures[\rmfamily]{Ligatures=TeX,Scale=1}
\fi
% Use upquote if available, for straight quotes in verbatim environments
\IfFileExists{upquote.sty}{\usepackage{upquote}}{}
\IfFileExists{microtype.sty}{% use microtype if available
  \usepackage[]{microtype}
  \UseMicrotypeSet[protrusion]{basicmath} % disable protrusion for tt fonts
}{}
\makeatletter
\@ifundefined{KOMAClassName}{% if non-KOMA class
  \IfFileExists{parskip.sty}{%
    \usepackage{parskip}
  }{% else
    \setlength{\parindent}{0pt}
    \setlength{\parskip}{6pt plus 2pt minus 1pt}}
}{% if KOMA class
  \KOMAoptions{parskip=half}}
\makeatother
\usepackage{xcolor}
\setlength{\emergencystretch}{3em} % prevent overfull lines
\setcounter{secnumdepth}{5}
% Make \paragraph and \subparagraph free-standing
\ifx\paragraph\undefined\else
  \let\oldparagraph\paragraph
  \renewcommand{\paragraph}[1]{\oldparagraph{#1}\mbox{}}
\fi
\ifx\subparagraph\undefined\else
  \let\oldsubparagraph\subparagraph
  \renewcommand{\subparagraph}[1]{\oldsubparagraph{#1}\mbox{}}
\fi

\usepackage{color}
\usepackage{fancyvrb}
\newcommand{\VerbBar}{|}
\newcommand{\VERB}{\Verb[commandchars=\\\{\}]}
\DefineVerbatimEnvironment{Highlighting}{Verbatim}{commandchars=\\\{\}}
% Add ',fontsize=\small' for more characters per line
\usepackage{framed}
\definecolor{shadecolor}{RGB}{241,243,245}
\newenvironment{Shaded}{\begin{snugshade}}{\end{snugshade}}
\newcommand{\AlertTok}[1]{\textcolor[rgb]{0.68,0.00,0.00}{#1}}
\newcommand{\AnnotationTok}[1]{\textcolor[rgb]{0.37,0.37,0.37}{#1}}
\newcommand{\AttributeTok}[1]{\textcolor[rgb]{0.40,0.45,0.13}{#1}}
\newcommand{\BaseNTok}[1]{\textcolor[rgb]{0.68,0.00,0.00}{#1}}
\newcommand{\BuiltInTok}[1]{\textcolor[rgb]{0.00,0.23,0.31}{#1}}
\newcommand{\CharTok}[1]{\textcolor[rgb]{0.13,0.47,0.30}{#1}}
\newcommand{\CommentTok}[1]{\textcolor[rgb]{0.37,0.37,0.37}{#1}}
\newcommand{\CommentVarTok}[1]{\textcolor[rgb]{0.37,0.37,0.37}{\textit{#1}}}
\newcommand{\ConstantTok}[1]{\textcolor[rgb]{0.56,0.35,0.01}{#1}}
\newcommand{\ControlFlowTok}[1]{\textcolor[rgb]{0.00,0.23,0.31}{#1}}
\newcommand{\DataTypeTok}[1]{\textcolor[rgb]{0.68,0.00,0.00}{#1}}
\newcommand{\DecValTok}[1]{\textcolor[rgb]{0.68,0.00,0.00}{#1}}
\newcommand{\DocumentationTok}[1]{\textcolor[rgb]{0.37,0.37,0.37}{\textit{#1}}}
\newcommand{\ErrorTok}[1]{\textcolor[rgb]{0.68,0.00,0.00}{#1}}
\newcommand{\ExtensionTok}[1]{\textcolor[rgb]{0.00,0.23,0.31}{#1}}
\newcommand{\FloatTok}[1]{\textcolor[rgb]{0.68,0.00,0.00}{#1}}
\newcommand{\FunctionTok}[1]{\textcolor[rgb]{0.28,0.35,0.67}{#1}}
\newcommand{\ImportTok}[1]{\textcolor[rgb]{0.00,0.46,0.62}{#1}}
\newcommand{\InformationTok}[1]{\textcolor[rgb]{0.37,0.37,0.37}{#1}}
\newcommand{\KeywordTok}[1]{\textcolor[rgb]{0.00,0.23,0.31}{#1}}
\newcommand{\NormalTok}[1]{\textcolor[rgb]{0.00,0.23,0.31}{#1}}
\newcommand{\OperatorTok}[1]{\textcolor[rgb]{0.37,0.37,0.37}{#1}}
\newcommand{\OtherTok}[1]{\textcolor[rgb]{0.00,0.23,0.31}{#1}}
\newcommand{\PreprocessorTok}[1]{\textcolor[rgb]{0.68,0.00,0.00}{#1}}
\newcommand{\RegionMarkerTok}[1]{\textcolor[rgb]{0.00,0.23,0.31}{#1}}
\newcommand{\SpecialCharTok}[1]{\textcolor[rgb]{0.37,0.37,0.37}{#1}}
\newcommand{\SpecialStringTok}[1]{\textcolor[rgb]{0.13,0.47,0.30}{#1}}
\newcommand{\StringTok}[1]{\textcolor[rgb]{0.13,0.47,0.30}{#1}}
\newcommand{\VariableTok}[1]{\textcolor[rgb]{0.07,0.07,0.07}{#1}}
\newcommand{\VerbatimStringTok}[1]{\textcolor[rgb]{0.13,0.47,0.30}{#1}}
\newcommand{\WarningTok}[1]{\textcolor[rgb]{0.37,0.37,0.37}{\textit{#1}}}

\providecommand{\tightlist}{%
  \setlength{\itemsep}{0pt}\setlength{\parskip}{0pt}}\usepackage{longtable,booktabs,array}
\usepackage{calc} % for calculating minipage widths
% Correct order of tables after \paragraph or \subparagraph
\usepackage{etoolbox}
\makeatletter
\patchcmd\longtable{\par}{\if@noskipsec\mbox{}\fi\par}{}{}
\makeatother
% Allow footnotes in longtable head/foot
\IfFileExists{footnotehyper.sty}{\usepackage{footnotehyper}}{\usepackage{footnote}}
\makesavenoteenv{longtable}
\usepackage{graphicx}
\makeatletter
\def\maxwidth{\ifdim\Gin@nat@width>\linewidth\linewidth\else\Gin@nat@width\fi}
\def\maxheight{\ifdim\Gin@nat@height>\textheight\textheight\else\Gin@nat@height\fi}
\makeatother
% Scale images if necessary, so that they will not overflow the page
% margins by default, and it is still possible to overwrite the defaults
% using explicit options in \includegraphics[width, height, ...]{}
\setkeys{Gin}{width=\maxwidth,height=\maxheight,keepaspectratio}
% Set default figure placement to htbp
\makeatletter
\def\fps@figure{htbp}
\makeatother
\newlength{\cslhangindent}
\setlength{\cslhangindent}{1.5em}
\newlength{\csllabelwidth}
\setlength{\csllabelwidth}{3em}
\newlength{\cslentryspacingunit} % times entry-spacing
\setlength{\cslentryspacingunit}{\parskip}
\newenvironment{CSLReferences}[2] % #1 hanging-ident, #2 entry spacing
 {% don't indent paragraphs
  \setlength{\parindent}{0pt}
  % turn on hanging indent if param 1 is 1
  \ifodd #1
  \let\oldpar\par
  \def\par{\hangindent=\cslhangindent\oldpar}
  \fi
  % set entry spacing
  \setlength{\parskip}{#2\cslentryspacingunit}
 }%
 {}
\usepackage{calc}
\newcommand{\CSLBlock}[1]{#1\hfill\break}
\newcommand{\CSLLeftMargin}[1]{\parbox[t]{\csllabelwidth}{#1}}
\newcommand{\CSLRightInline}[1]{\parbox[t]{\linewidth - \csllabelwidth}{#1}\break}
\newcommand{\CSLIndent}[1]{\hspace{\cslhangindent}#1}

\KOMAoption{captions}{tableheading}
\makeatletter
\makeatother
\makeatletter
\@ifpackageloaded{bookmark}{}{\usepackage{bookmark}}
\makeatother
\makeatletter
\@ifpackageloaded{caption}{}{\usepackage{caption}}
\AtBeginDocument{%
\ifdefined\contentsname
  \renewcommand*\contentsname{Table of contents}
\else
  \newcommand\contentsname{Table of contents}
\fi
\ifdefined\listfigurename
  \renewcommand*\listfigurename{List of Figures}
\else
  \newcommand\listfigurename{List of Figures}
\fi
\ifdefined\listtablename
  \renewcommand*\listtablename{List of Tables}
\else
  \newcommand\listtablename{List of Tables}
\fi
\ifdefined\figurename
  \renewcommand*\figurename{Figure}
\else
  \newcommand\figurename{Figure}
\fi
\ifdefined\tablename
  \renewcommand*\tablename{Table}
\else
  \newcommand\tablename{Table}
\fi
}
\@ifpackageloaded{float}{}{\usepackage{float}}
\floatstyle{ruled}
\@ifundefined{c@chapter}{\newfloat{codelisting}{h}{lop}}{\newfloat{codelisting}{h}{lop}[chapter]}
\floatname{codelisting}{Listing}
\newcommand*\listoflistings{\listof{codelisting}{List of Listings}}
\makeatother
\makeatletter
\@ifpackageloaded{caption}{}{\usepackage{caption}}
\@ifpackageloaded{subcaption}{}{\usepackage{subcaption}}
\makeatother
\makeatletter
\@ifpackageloaded{tcolorbox}{}{\usepackage[many]{tcolorbox}}
\makeatother
\makeatletter
\@ifundefined{shadecolor}{\definecolor{shadecolor}{rgb}{.97, .97, .97}}
\makeatother
\makeatletter
\makeatother
\ifLuaTeX
  \usepackage{selnolig}  % disable illegal ligatures
\fi
\IfFileExists{bookmark.sty}{\usepackage{bookmark}}{\usepackage{hyperref}}
\IfFileExists{xurl.sty}{\usepackage{xurl}}{} % add URL line breaks if available
\urlstyle{same} % disable monospaced font for URLs
\hypersetup{
  pdftitle={Impact des aires protégées sur la déforestation : guide de formation pratique},
  pdfauthor={Florent Bédécarrats, Marc Bouvier, Kenneth Houngbedji; Jeanne de Montalembert et Marin Ferry},
  colorlinks=true,
  linkcolor={blue},
  filecolor={Maroon},
  citecolor={Blue},
  urlcolor={Blue},
  pdfcreator={LaTeX via pandoc}}

\title{Impact des aires protégées sur la déforestation : guide de
formation pratique}
\author{Florent Bédécarrats, Marc Bouvier, Kenneth Houngbedji; Jeanne de
Montalembert et Marin Ferry}
\date{10/3/22}

\begin{document}
\maketitle
\ifdefined\Shaded\renewenvironment{Shaded}{\begin{tcolorbox}[sharp corners, borderline west={3pt}{0pt}{shadecolor}, enhanced, frame hidden, breakable, interior hidden, boxrule=0pt]}{\end{tcolorbox}}\fi

\renewcommand*\contentsname{Table of contents}
{
\hypersetup{linkcolor=}
\setcounter{tocdepth}{2}
\tableofcontents
}
\bookmarksetup{startatroot}

\hypertarget{pruxe9face}{%
\chapter*{Préface}\label{pruxe9face}}
\addcontentsline{toc}{chapter}{Préface}

\markboth{Préface}{Préface}

Ce contenu a été développé afin de servir de support pédagogique pour
l'atelier ``évaluation des politiques'' de la session 2022 des
Universités en sciences sociales Tany Vao. Les universités Tany Vao
visent à dispenser une formation à la recherche de haut niveau à
l'attention de doctorants et jeunes chercheurs de Madagascar et
d'Afrique de l'Ouest. Après deux jours de plénières, les participants se
répartissent pendant cinq jours entre quatre ateliers parallèles :
socioéconomie, éthnoégologie, anthropologie et évaluation des
politiques.

L'atelier ``évaluation des politiques'' adopte une approche axée
l'économétrie et la science des données. Il alterne des sessions
théorique et pratique. Conformément au thème phare de Tany Vao pour 2022
(``environnement et sociétés''), le cas d'étude choisi pour servir de
fil rouge à ces travaux est l'impact des aires protégées sur la
déforestation.

\bookmarksetup{startatroot}

\hypertarget{introduction}{%
\chapter{Introduction}\label{introduction}}

\hypertarget{outils-utilisuxe9s}{%
\section{Outils utilisés}\label{outils-utilisuxe9s}}

Les éléments ci-dessous constituent le support pour les sessions
pratiques de cet atelier. Ils sont réalisés en suivant une approche
ouverte et reproductible fondée sur un document de type ``notebook''
(\textbf{bedecarrats\_alternative\_2017?}). Un notebook rassemble à la
fois :

\begin{itemize}
\item
  les lignes de code du programme statistique qui traite les données ;
\item
  les résultats (calculs, tableaux, graphiques\ldots) produits lors de
  l'exécution de ce programme ;
\item
  le texte rédigé par les auteurs pour expliquer le processus d'analyse
  et en interpréter les résultats.
\end{itemize}

L'intérêt du format notebook, par rapport à l'utilisation de documents
distincts pour traiter les données d'une part, et en analyser les
résultats d'autre part, est multiple :

\begin{itemize}
\item
  favoriser la reproductibité de la recherche (tout le processus de
  traitement, analyse, interprétation peut être inspecté et dupliqué) ;
\item
  faciliter le travail du chercheur (une interface pour tout faire) ; et
\item
  assurer les meilleures pratiques de collaboration (utilisation pour le
  versionnage, partage et fusion des travaux les outils performants
  développés en programmation informatique).
\end{itemize}

Les traitements sont réalisés en R, qui est à la fois un logiciel et un
langage open sources dédiés à l'analyse de données. Les traitements R
sont inclus dans un document Quarto, un format qui exécute aussi bien
des codes en R, Python, e rendus dans différents formats (LaTeX/PDF,
HTML ou Word).

On s'appuie sur le package R \{mapme.biodiversity\}, développé par la
KfW dans le cadre de l'initiative commune MAPME qui associe la KfW et
l'AFD. Le package \{mapme.biodiversity\} facilite l'acquisition et la
préparation d'un grand nombre de données (CHIRPS, Global Forest Watch,
FIRMS, SRTM, Worldpop\ldots) et calculer un grand nombre d'indicateurs
de manière harmonisée (active\_fire\_counts, biome classification, land
cover classification, population count, precipitation, soil properties,
tree cover loss, travel time\ldots). Une documentation riche est
disponible sur le portail dédié du package en question (Kluve et al.
2022).

On mobilise aussi les codes d'analyse d'impact développés par la même
équipe et mises à disposition dans le dépôt Github:
\url{https://github.com/openkfw/mapme.protectedareas}. Le code développé
par l'équipe est assez complexe. A des fins pédagogiques et pour
s'assurer qu'on l'a bien compris, on propose ici une version simplifiée
(en cours de développement).

Les sources pour l'ensemble du code source et du texte du présent
document est accessible sur Github à l'adresse suivante :
\url{https://github.com/fBedecarrats/deforestation_madagascar}. Les
analyses sont menées sur la plateforme SSP Cloud, mises à disposition
par l'INSEE pour les data scientists travaillant pour des
administrations publiques. Il s'agit d'une instance de stockage de
données massif (S3) et de calcul haute performance (cluster Kubernetes)
disposant d'une interface simplifiée permettant à l'utilisateur de
configurer, lancer et administrer facilement des environnements de
traitement de données (RStudio server, Jupyter lab ou autres\ldots). Le
code est conçu pour s'exécuter de la même manière en local sur un PC,
mais la préparation des données sera certainement beaucoup plus longue à
exécuter.

\begin{Shaded}
\begin{Highlighting}[]
\CommentTok{\# \# Le package est en cours de développement, toujours installer la version en cours}
\CommentTok{\# remotes::install\_github("mapme{-}initiative/mapme.biodiversity", }
\CommentTok{\#                         upgrade = "always")}

\NormalTok{librairies\_requises }\OtherTok{\textless{}{-}} \FunctionTok{c}\NormalTok{( }\CommentTok{\# On liste les librairies dont on a besoin}
  \StringTok{"dplyr"}\NormalTok{, }\CommentTok{\# Pour faciliter la manipulation de données tabulaires}
  \StringTok{"readxl"}\NormalTok{, }\CommentTok{\# Pour lire les fichiers excel (Carvalho et al. 2018)}
  \StringTok{"tidyr"}\NormalTok{, }\CommentTok{\# Pour reformater les données (pivots...)}
  \StringTok{"stringr"}\NormalTok{, }\CommentTok{\# Pour manipuler des chaînes de caractères}
  \StringTok{"ggplot2"}\NormalTok{, }\CommentTok{\# Pour faire des graphiques}
  \StringTok{"cowplot"}\NormalTok{, }\CommentTok{\# Pour arranger des graphiques en illustrations composées}
  \StringTok{"gt"}\NormalTok{, }\CommentTok{\# Pour des rendus graphiques harmonisés html et pdf/LaTeX}
  \StringTok{"lubridate"}\NormalTok{, }\CommentTok{\# Pour manipuler des dates}
  \StringTok{"sf"}\NormalTok{, }\CommentTok{\# Pour faciliter la manipulation de données géographiques}
  \StringTok{"wdpar"}\NormalTok{, }\CommentTok{\# Pour télécharger simplement la base d\textquotesingle{}aires protégées WDPA}
  \StringTok{"webdriver"}\NormalTok{, }\CommentTok{\# requis pour installer phantomjs pour wdpar}
  \StringTok{"tmap"}\NormalTok{, }\CommentTok{\# Pour produire de jolies carte}
  \StringTok{"geodata"}\NormalTok{, }\CommentTok{\# Pour télécharger simplement les frontières administratives}
  \StringTok{"tidygeocoder"}\NormalTok{, }\CommentTok{\# pour obtenir les coordo GPS d\textquotesingle{}un point à partir de son nom}
  \StringTok{"maptiles"}\NormalTok{, }\CommentTok{\# Pour télécharger des fonds de carte }
  \StringTok{"purrr"}\NormalTok{, }\CommentTok{\# Pour utiliser des formes fonctionnelles de programmation (ex. map)}
  \StringTok{"mapme.biodiversity"}\NormalTok{, }\CommentTok{\# Acquisition et traitement des données du projet}
  \StringTok{"plm"}\NormalTok{, }\CommentTok{\# Linear Models for Panel Data and robust covariance matrices}
  \StringTok{"broom"}\NormalTok{, }\CommentTok{\# pour reformater simplement les rendus de tests statistiques}
  \StringTok{"stargazer"}\NormalTok{, }\CommentTok{\# Reformater de manière plus lisible les résumé des régressions}
  \StringTok{"MatchIt"}\NormalTok{, }\CommentTok{\# Pour le matching}
  \CommentTok{\#"glm", \# Modèles linéaires généralisés (pour le PSM)}
  \StringTok{"optmatch"}\NormalTok{, }\CommentTok{\# Fonctions d\textquotesingle{}optimisation du matching}
  \StringTok{"cobalt"}\NormalTok{) }\CommentTok{\# Tables et graphs d\textquotesingle{}équilibre des groupes de matching}
  
\CommentTok{\# On regarde parmi ces librairies lesquelles ne sont pas installées}
\NormalTok{manquantes }\OtherTok{\textless{}{-}} \SpecialCharTok{!}\NormalTok{(librairies\_requises }\SpecialCharTok{\%in\%} \FunctionTok{installed.packages}\NormalTok{())}
\CommentTok{\# On installe celles qui manquent}
\ControlFlowTok{if}\NormalTok{(}\FunctionTok{any}\NormalTok{(manquantes)) }\FunctionTok{install.packages}\NormalTok{(librairies\_requises[manquantes])}
\CommentTok{\# On charge toutes les librairies requises}
\FunctionTok{invisible}\NormalTok{(}\FunctionTok{lapply}\NormalTok{(librairies\_requises, require, }\AttributeTok{character.only=} \ConstantTok{TRUE}\NormalTok{))}

\CommentTok{\# Système de coordonnées géographiques utilisées pour le projet : EPSG:29739}
\NormalTok{mon\_scr }\OtherTok{\textless{}{-}} \StringTok{"EPSG:29739"} \CommentTok{\# correspondant à Tananarive / UTM zone 39S}
\CommentTok{\# Surface des hexagones en km2}
\NormalTok{taille\_hex }\OtherTok{\textless{}{-}} \DecValTok{5}
\CommentTok{\# Taille des titres des cartes}
\NormalTok{taille\_titres\_cartes }\OtherTok{=} \DecValTok{1}
\CommentTok{\# on crée un dossier de données si pas déjà disponible}
\FunctionTok{dir.create}\NormalTok{(}\StringTok{"data"}\NormalTok{)}
\CommentTok{\# Désactiver les notations scientifiques}
\FunctionTok{options}\NormalTok{(}\AttributeTok{scipen =}\DecValTok{999}\NormalTok{)}
\end{Highlighting}
\end{Shaded}

\hypertarget{mode-demploi}{%
\section{Mode d'emploi}\label{mode-demploi}}

\bookmarksetup{startatroot}

\hypertarget{summary}{%
\chapter{Summary}\label{summary}}

In summary, this book has no content whatsoever.

\bookmarksetup{startatroot}

\hypertarget{references}{%
\chapter*{References}\label{references}}
\addcontentsline{toc}{chapter}{References}

\markboth{References}{References}

\hypertarget{refs}{}
\begin{CSLReferences}{1}{0}
\leavevmode\vadjust pre{\hypertarget{ref-kluve_kfw_2022}{}}%
Kluve, Jochen, Johannes Schielain, Melvin Wong, and Yota Eilers. 2022.
{``The {KfW} {Protected} {Areas} {Portfolio}: A {Rigorous} {Impact}
{Evaluation}.''} Frankfürt.

\end{CSLReferences}



\end{document}
